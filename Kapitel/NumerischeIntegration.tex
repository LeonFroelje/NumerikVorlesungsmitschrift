\documentclass[11pt,a4paper,oneside]{scrartcl}

\usepackage{amsmath}
\usepackage{amsthm}
\usepackage{amssymb}
\usepackage[utf8]{inputenc}
\usepackage{amsfonts}
\usepackage{phaistos}
\usepackage{enumerate}
\usepackage{mathtools}
\usepackage[ngerman]{babel}
\usepackage{xparse}
\usepackage{graphicx}
\usepackage{amsthm}
\usepackage{wasysym}
\usepackage{fontawesome}
\usepackage{marvosym}
\usepackage{pdfpages}
\usepackage{color}
\usepackage[T1]{fontenc}
\usepackage{textcomp}
\usepackage{gensymb}
\usepackage{newunicodechar}
\graphicspath{{./Bilder/}} 
\setlength{\parindent}{0em}
\usepackage[center]{caption}
\usepackage[normalem]{ulem}
\usepackage{booktabs} 
\usepackage{subfig} 
\usepackage{float}
\usepackage{xpatch}
\usepackage[shortlabels]{enumitem}
\usepackage[textwidth=13cm]{geometry}
\usepackage[notref,notcite]{showkeys}
\usepackage{soul}
\usepackage{mathrsfs}
\usepackage{tikz}

\newcommand{\dsum}{\displaystyle\sum}
\newcommand{\K}{\mathbb{K}}
\newcommand{\Z}{\mathbb{Z}}
\newcommand{\Q}{\mathbb{Q}}
\newcommand{\R}{\mathbb{R}}
\newcommand{\N}{\mathbb{N}}
\newcommand{\C}{\mathbb{C}}
\newcommand{\M}{M\text{°}}
\newcommand{\calM}{\ensuremath{\mathcal{M}}}
\newcommand{\calMnm}{\ensuremath{\calM_{n\times m}}}
\newcommand{\calMn}{\ensuremath{\calM_{n}}}
\newcommand{\lin}[1]{\ensuremath{\text{span\ensuremath{\left\{ #1 \right\}}}}}
\newcommand{\Kern}[1]{\ensuremath{\text{Ker}\left(#1\right)}}
\newcommand{\func}[2]{\ensuremath{#1 \left(#2\right)}}
\newcommand{\Bild}[1]{\ensuremath{\text{Bild}\left(#1\right)}}
\newcommand{\rang}[1]{\ensuremath{\text{Rang}\left(#1\right)}}
\newcommand{\F}{\mathbb{F}}
\newcommand{\bbP}{\mathbb{P}}
\newcommand{\toInfty}{\ensuremath{\lim_{n\to\infty}}}
\newcommand{\RN}[1]{\textup{\uppercase\expandafter{\romannumeral#1}}}
\newcommand{\matrx}[1]{\begin{pmatrix}#1\end{pmatrix}}
\newcommand{\set}[1]{\ensuremath{\left\{ #1 \right\}}}
\newcommand{\abs}[1]{\ensuremath{\left\vert #1 \right\vert}}
\newcommand{\dabs}[1]{\ensuremath{\left\vert\left\vert#1\right\vert\right\vert}}
\newcommand{\scp}[2]{\ensuremath{\left<#1,\smallspace #2\right>}}
\newcommand{\dbcup}[2][]{\ensuremath{\displaystyle\bigcup_{#1}^{#2}}}
\newcommand{\dbcap}[2][]{\ensuremath{\displaystyle\bigcap_{#2}^{#1}}}
\newcommand{\cconj}[1]{\overline{#1}}
\newcommand{\sym}{\ensuremath{\text{Sym }}}
\newcommand{\midspace}{\hspace{.5em}}
\newcommand{\smallspace}{\hspace{.2em}}
\newcommand{\largespace}{\hspace{1em}}
\newcommand{\txt}[1]{\ensuremath{\text{#1}}}
\newcommand{\dprod}{\ensuremath{\displaystyle\prod}}
\newcommand{\eps}{\varepsilon}
\def\igb#1#2#3{\ensuremath{#1 = #2,\dots\if \empty#3\else, #3\fi}}
\newcommand*\diff{\mathop{}\!\mathrm{d}}
\newcommand{\tjadosize}{\tiny}
\newcommand{\Tjadosize}{\Huge}
\def\Cock#1{%
    \def\temp{#1}\ifx\temp\empty{%
        \begin{tikzpicture}
            \draw[ultra thick] (0,0) circle (1);%
            \draw[ultra thick] (0,2) circle(1);%
            \draw[ultra thick] (1,0) -- (7,0);%
            \draw[ultra thick] (1,2) -- (7,2);%
            \draw[ultra thick] (7,-0.2) -- (7,2.2);%
            \draw[ultra thick] (8.2,1) arc (0:-90:1.2);%
            \draw[ultra thick] (8.2,1) arc(0:90:1.2);%
            \draw[ultra thick] (8.2,1) -- (7.7,1); % Pissschlitz
        \end{tikzpicture}%
    }%
    \else{%
    \begin{tikzpicture}
        \draw[ultra thick] (0,0) circle (1);
        \draw[ultra thick] (0,2) circle(1);
        \draw[ultra thick] (1,0) -- (#1,0);
        \draw[ultra thick] (1,2) -- (#1,2);
        \draw[ultra thick] (#1,-0.2) -- (#1,2.2);
        \draw[ultra thick] (#1 +1.2,1) arc (0:-90:1.2);
        \draw[ultra thick] (#1 +1.2,1) arc(0:90:1.2);
        \draw[ultra thick] (#1 +1.2,1) -- (#1 +0.7,1); % Pissschlitz
    \end{tikzpicture}
    }
    \fi
}
\def\Cocks#1{
    \newcount\cockcount
    \loop\ifnum\cockcount < #1 \Cock{}\\\advance\cockcount by 1 \repeat \cockcount = 0
}
\def\Cockcum{
    \rotatebox{0}{
        \begin{tikzpicture}
            \draw[ultra thick] (0,0) circle (1);
            \draw[ultra thick] (0,2) circle(1);
            \draw[ultra thick] (1,0) -- (7,0);
            \draw[ultra thick] (1,2) -- (7,2);
            \draw[ultra thick] (7,-0.2) -- (7,2.2);
            \draw[ultra thick] (8.2,1) arc (0:-90:1.2);
            \draw[ultra thick] (8.2,1) arc(0:90:1.2);
            \draw[ultra thick] (8.2,1) -- (7.7,1); % Pissschlitz
            \draw[very thick] (8.3 , 1) -- (9, 1);
            \draw[very thick] (9.3, 1) to[bend left=10] (10, 0.8);
            \draw[very thick] (10.3, 0.6) to [bend left=20] (10.7, 0.2);
            \draw[very thick] (10.8, -0.2) to [bend left=10] (10.9, -0.7);
            \draw[very thick] (10.9, -1) to (10.9, -1.5);
            \fill[color=black!50, ultra thick] (10.9, -2.5) circle(0.6);
            \draw[ultra thick] (10.9, -2.5) circle(0.62);
        \end{tikzpicture}
    }
}
\def\YoOli{
    Check
    Young Alpha, auf entspannt, ja
    Heut ist ein guter Tag, sie weiß, was ich mag
    Heut ist ein guter Tag
    Dass ihr macht, was ich sag'
    Ist, was ich an euch mag
    Sarah, Lara und Tamara
    Sind immer startklar
    Ich baller' mehr als Osama
    Sie lecken an mir, als wär ich ein Lollipop
    Doch sie sind nicht mehr zwölf
    Sie sind über achtzehn
    Deswegen gibt es meinen Cock
    Frisch, einen frischen Penis
    Es ist der einzig Wahre und der Lange
    Nicht dieser Kleine, nein, der Feine
    Er ist frisch rasiert mit zwei Eiern
    Sie machen, was ich will und sucken an mir
    Weil sie mich feiern
    Wow-wow-wow-wow-wow, neues \Telefon
    IPhone, iPhone, iPhone 11 Pro
    Wow-wow-wow-wow-wow, das sind keine Hoes
    Ich lieb' sie aus ganzem Herzen
    Sie komm'n von allein
    Sie stöhn'n und müssen laut schrei'n
    Sie komm'n von allein
    Sie stöhn'n und müssen laut schrei'n
    Sie mögen es hart also beiß' ich in die Titten rein
    Sie wollen's hart ich beiß' in die Titten rein
    Heut ist ein guter Tag
    Dass ihr macht, was ich sag'
    Ist, was ich an euch mag
    Sarah, Lara und Tamara
    Sind immer startklar
    Ich baller' mehr als Osama
    Die drei lecken an mei'm Lollipop
    Und sie kennen kein Stop, kennen kein Stop
    Den ganzen Tag, ganze Nacht
    Doch ich hab noch, doch ich hab noch anderes im Kopf vor
    An mei'm Lollipop, das ist Hip-Hop
    Und bisschen Pop mit bisschen Pop
    Ja, wenn du mich schon liebst
    Dann kannst du auch mein Rücken kraul'n, Rücken kraul'n
    Ja, wenn mich so liebst
    Kannst du vier Stunden mein'n Rücken kraul'n
    Mein'n Rücken kraul'n
    Und du weißt, ich bin lieb, auch, wenn ich dir dein'n Popo hau'
    Auf den Popo hau'
    Sie komm'n von allein
    Sie stöhn'n und müssen laut schrei'n
    Sie komm'n von allein
    Sie stöhn'n und müssen laut schrei'n
    Sie mögen es hart also beiß' ich in die Titten rein, ah
    Sie wollen's hart ich beiß' in die Titten rein
    Heut ist ein guter Tag
    Dass ihr macht, was ich sag'
    Ist, was ich an euch mag
    Sarah, Lara und Tamara
    Sind immer startklar
    Ich baller' mehr als Osama
    Young Alpha
}
    
\newcount\loopcounter
\newcommand{\Schwamms}[1]{8\loop\ifnum\loopcounter < #1 =\advance\loopcounter by 1 \repeat D \loopcounter=0}

\renewcommand{\ln}[1]{\ensuremath{\text{ln}\left(#1\right)}}
\renewcommand{\det}[1]{\ensuremath{\text{det}\left(#1\right)}}
\def\cos#1{\ensuremath{\text{cos}\if \empty#1\else\left(#1\right)\fi}}
\def\sin#1{\ensuremath{\text{sin}\if \empty#1\else\left(#1\right)\fi}}
\def\calO#1{\ensuremath{\mathcal{O}\if \empty#1\else\left(#1\right)\fi}}
\def\deg#1{\ensuremath{\text{deg}\if \empty#1\else\left(#1\right)\fi}}
\def\Pol#1{\ensuremath{\if \empty#1\bbP \else\bbP_{#1}\fi}}

\renewenvironment{proof}{{\bfseries Beweis}}{\qed\newline}

\theoremstyle{definition}
\newtheorem{theorem}{Satz}[section]
\newtheorem{corollary}[theorem]{Korollar}
\newtheorem{lemma}[theorem]{Lemma}
\newtheorem{definition}[theorem]{Definition}
\newtheorem{example}[theorem]{Beispiel}
\newtheorem{remark}[theorem]{Bemerkung}
\newtheorem{proposition}[theorem]{Proposition}
\newtheorem*{question}{Frage}
\newtheorem*{reminder}{Erinerung}
\newtheorem*{remindexample}{Erinnerung/Beispiel}
\newtheorem*{motivation}{Motivation}

\begin{document}
\section{Numerische Integration}
Berechnung von Integralen, z.B. zur Flächen- oder Volumenberechnung, 
aber auch 
notwendig in komplexeren Formeln/Algorithmen, z.B. Fourier-Integrale, 
Numerik 
partieller-Differentialgleichungen. Oft nicht (leicht) von Hand zu 
berechnen, 
\(\Rightarrow\) Algorithmen zur näherungsweisen Berechnung von 
Integralen.\\
Viele typiche ''\underline{Quadraturformeln}'' haben für \(f\in C[a,b]\)
die Form 
\[
\int^b_a f(x)\, dx \approx \sum_{i=0}^n x_i f(x_i),
\]
 d.h. Kombination von Punktauswertungen mit Stützstellen \( a\leq x_0 < 
 x_1\dots 
 x_n\leq b\)
\begin{remindexample}[Rieman-Integral]\hfill\\
z.B. Rieman-Summe \[
I_h(f)\coloneqq \sum_{i=1}^n f(x_{i-1})\cdot(x_i-x_{i-1})
\]
\(f\) Rieman-Integrierbar \(\leadsto I_h(f)\to I(f) \text{ für } h\to 
0, \ 
h\coloneqq \max(x_i-x_{i-1})\)\\
\end{remindexample}
\subsection{Interpolatorische Quadraturformel}
Kennt man eine Polynom-Interpolation von $f$, (oder Hermite-), kann man 
statt $f$ 
einfach die Interpolierende integrieren. Integration über Polynome ist 
einfach.
Zu \(a\leq x_0 <x_1\dots < x_n\leq b\) sei \(P_n\in\Pol_n\) das 
interpolierende 
Polynom zu \(f\) mit \(P_n(x_i)=f(x_i), i=0,\dots, n\)\\
setze dann \[
I^{(n)}(f)\coloneqq \int^b_aP_n(x)\, dx =\int^b_a \sum_{i=0}^n 
f(x_i)L^{(n)}_i(x)\, dx = \sum_{i=0}^n f(x_i)\int^b_a L^{(n)}_i(x)\, dx
\]
\[
P_n(x)=\sum_{i=0}^n f(x_i)L^{(n)}_i(x)
\]
Wie groß ist der Fehler \(I(f)-I^{(n)}\)?\\
Mit der Formel fürden Interpolationen Fehler folgt:

\begin{theorem}
    Für die Lagrange-Quadraturformel $I^{(n)}$ gilt, falls $f \in
    C^{n+1}[a,b]$:\[
        I(f) - I^{(n)}(f) = \int_a^b f(x) - p_n(x) dx = \int_a^b\frac{1}
        {(n+1)!}
        f^{(n+1}(\xi_x)\prod_{j=0}^n(x-x_j)dx
    \]
    also \[   
        \abs{I(f) - I^{(n)}(f)} \leq \frac{1}{(n+1)!}\cdot 
        \max\limits_{[a,b]}
        \abs{f^{(n+1)}}\cdot \abs{\int_a^b\prod_{j=0}^n(x-x_j) dx}
    \]
\end{theorem}
\begin{remark}
    man kann auch zeigen:\[
    I(f) - I^{(n)}(f) = \int_a^b f[x_0, \dots, x_n, x] \prod_{j=0}^n(x-
    x-j)dx
    \] 
\end{remark}

Interpolatorische Integrationsformeln, \(I^{(n)}
(f)\coloneqq\int^b_ap_n(x)\diff 
x\), \( p_n\in\Pol\) des Interpolationspolynoms zu \(f\) in  
\(x_0,\dots,x_n
\in[a,b]\) Nach Konstruktion: die interpolierende Quadraturformel ist 
''exakt'' 
für beliebige Polynome \(p\in\Pol_n\), wegen des Eindeutigkeit der 
Interpolationspolynoms.
\begin{definition}
    Eine Quadraturformel \(I^{(n)}\) wird (mindestens) ''\underline{von 
    der 
    Ordnung m}'' genannt, falls durch sie alle Polynomevom Grad \(\leq 
    m-n\) 
    exakt integriert werden.
\end{definition}
Damit sind die interpolatorischen Quadraturformeln \(I^{(n)}\) 
mindestens von der 
Ordnung \(n+1\).

\begin{example}[Lagrange-Quadraturen mit n+1 Stützstellen mit gleichen 
Abständen]\hfill
\begin{enumerate}[(a)]
    \item ''(abgeschlossene) \underline{Newton-Cotes-Formeln}'':\\
    $a,b$ sind Stützstellen, $x_i=a+ih,\midspace \igb i 0 n$ mit 
    $h=\frac{b-a}{n}$
    \item ''\underline{offene Newton-Cotes-Formeln}:\\
    $a,b$ sind \underline{keine} Stützstellen, $x_i = a+(i+1)h,\midspace
    \igb i 0 
    n, \midspace h=\frac{b-a}{n+2}$
\end{enumerate}
Die ersten Newton-Cotes-Formeln sind:
\begin{description}
\item[abgeschlossen:]\begin{align*}
    I^{(1)} &= \frac{b-a}{2}(f(a) + f(b)) && 
    \text{''\underline{Trapezregel}''}\\
    I^{(2)} &= \frac{b-a}{6}\left(f(a) + 4f\left(\frac{ab}{2}\right) + 
    f(b)\right) && \text{''\underline{Keplersche Fassregel}''}\\
    I^{(3)} &= \frac{b-a}{8}\left(f(a) + 3f(a+h) + 3f(b-h) + f(b)\right)
    && 
    \text{''\underline{$\frac{3}{8}$-Regel}''}
\end{align*}    
\item[offen:]\begin{align*}
    I^{(0)}(f) &\coloneqq (b-a) \cdot f\left(\frac{a+b}{2}\right) && 
    \text{''\underline{Mittelpunktsregel}''}\\
    I^{(1)}(f) &\coloneqq \frac{b-a}{2}(f(a+h) + f(b-h))\\
    I^{(2)}(f) &\coloneqq \frac{b-a}{3}\left(2f(a+h) - f\left(\frac{a+b}
    {2}\right) + 2f(b-h)\right)  
\end{align*}

\end{description}
\end{example}
Mit den Interpolationsabschätzungen und Integral-Mittelwertsätzen zeigt 
man:

\begin{theorem}[Quadraturfehler Newton-Cotes-Formeln]\hfill
    \begin{enumerate}[i)]
        \item Für die Trapezregel \(I^{(1)}\) mit \(f\in C^2[a,b]\) 
        gilt: \[
        \int^b_af(x)\diff x-\frac{b-a}{2}(f(a)+f(b)) = -\frac{(b-
        a)^3}{12}f''(\xi) \text{ mit } \xi\in[a,b]
        \]
        \item Für die Simpson-Regel \(I^{(2)}\) mit \(f\in C^4[a,b]\) 
        gilt: \[
        \int^b_af(x)\diff x-\frac{b-a}{2}(f(a)+4f(\frac{a+b}{2})+f(b)) 
        = 
        -\frac{(b-a)^5}{2880}f^{(4)}(\xi) \text{ mit } \xi\in[a,b]
        \]
        \item Für die Mittelpunktformel \(I^{(0)}\) mit \(f\in 
        C^2[a,b]\) gilt: \[
        \int^b_af(x)\diff x-(b-a)f(\frac{a+b}{2} = -\frac{(b-
        a)^3}{24}f^{(2)}(\xi) \text{ mit } \xi\in[a,b]
        \]
    \end{enumerate}
\end{theorem}
\begin{proof}
    zu i) \begin{align*}
    \int^b_af(x) \diff x - \int^b_a p_n(x) \diff x &= \int^b_a f(x)-
    p_n(x) \diff 
    x\\
    &=\int^b_a f''(\xi(x))\cdot \frac{1}{2}\cdot(x-a)(x-b)\diff x\\
    &= f''(\xi)\frac{1}{2}\int^b_a(x-a)(x-b)\diff x\\
    &= \frac{1}{12}f''(\xi)(b-a)^3
   \end{align*}
\end{proof}
\begin{remark}
    zu iii) Mittelpunktformel ist exakt nicht nur für \(p\in\Pol_0\) 
    sondern 
    sogar für alle \(p\in\Pol_n\)
\end{remark}

\begin{remark}
    Sind neben $f$ auch die Ableitungen $f'(x)$ bekannt, dann kann man 
    auch eine 
    Hermite-Interpolation zur Herleitung von Quadraturformeln nehmen, 
    die Hermite-
    Interpolationsfehlerabschätzung überträgt sich dann auf die 
    Quadraturfehlerabschätzung.
\end{remark}
Um ein Integral besser zu approximieren, wird typicherweise nicht der 
Polynomgrad 
weiter erhöht, sondern eine Quadraturformel mit relativ geringen Grad 
auf 
Teilintervallen immer kleinerer Größe genutzt:\\
z.B.    $a = y_0 < y_1 < \dots < y_N = b$ mit Teilintervallen $I_i = 
[y_{i-1}, 
    y_i]$:\[
        \int_a^b f(x)\diff x = \sum_{i=1}^N \int_{I_i}f(x)\diff x\approx
        \underset{I_h^{(n)}(f)}{\underbrace{\dsum_{i=1}^N \int_{I_i} 
        \underset{\text{Interpolierende}}
        {\underbrace{I^{(n)}_{[y_{i-1}, y_i]}f}}\diff x}}
    \] 
    und \begin{align*}
                \int_a^b f(x)\diff x - I_h^{(n)}(f) &= \dsum_{i=1}^N 
                \int_{I_i}f(x)\diff x - \int_a^b\left(I_{[y_{i-1}, 
                y_i]}^{(n)}f\right)(x)\diff x\\
                &\leq \dsum_{i=1}^N c_m\cdot(y_{i-1}-
                y_i)^{m+1}\abs{f^{(m)}
                (\xi_i)}\\
                &\leq \dsum_{i=1}^N c_mh^{m+1}\cdot 
                \dabs{f^{(m)}}_{\max[a,b]}\\
                &\leq c_m(b-a)\frac{h}
                {h_{\min}}h^{m}\cdot\dabs{f^{(m)}}_{\max}
    \end{align*}
    mit \[
        N\leq \frac{b-a}{h_{\min}} , \largespace \frac{h}{h_{\min}}= 1, 
        \text{ 
        falls 
        alle Teilintervalle gleich lang sind}
    \]
\begin{example}
    $y_i = a+ih, \largespace h=\frac{b-a}{N}, \largespace \igb i 0 N$ 
    gleichgroße 
    Teilintervalle. 
    Summierte Trapezregel \begin{align*}
        I^{(1)}_h(f)\coloneqq \frac{h}{2}\left(f(a)+ \dsum_{i=1}^{N-
        1}2f(y_i)+f(b)\right), \\
        \abs{I(f)-I^{(1)}_h(f)}\leq \frac{b-a}{12}h^2
        \abs{\abs{f^{(2)}}}_{\max[a,b]}
    \end{align*}
    Summierte Simpsonregel: \begin{align*}
        I^{(2)}_h(f)\coloneqq \frac{h}{6}\left(f(a)+ \dsum_{i=1}^{N-
        1}2f(y_i)+\dsum_{i=1}^{N}4f\middle(\frac{y_{i-1}+y_i}
        {2}+f(b)\middle)
        \right), \\
        \abs{I(f)-I^{(2)}_h(f)}\leq \frac{b-a}{2880}h^4
        \abs{\abs{f^{(4)}}}_{\max}
    \end{align*}
    Summierte Mittelpunktregel: \begin{align*}
        I^{(0)}_h(f)\coloneqq h\dsum_{i=1}^{N}f\left(\frac{y_{i-1}+y_i}
        {2}\right), \\
        \abs{I(f)-I^{(0)}_h(f)}\leq \frac{b-a}{24}h^2
        \abs{\abs{f^{(2)}}}_{\max}
    \end{align*}
\end{example}
\begin{remark}
    Ähnlich geht es für Hermite-Splines, d.h. lokale Hermite-
    Interpolierende
\end{remark}
\begin{motivation}
    Mittelpunktsregel und Simpson-Regel sind von höherer Ordnung als 
    man es durch 
    den Polynomgrad alleine erwarten würde, anscheinend allein durch 
    die 
    geschickte Wahl der Stützstellen. 
\end{motivation}
\begin{question}
    Wie gut kann man werden bei optimaler Wahl der Stützstellen?
\end{question}
\subsection{Gauß-Quadraturformeln}
Man sieht leicht, dass die Maximale Ordnung einer interpolierenden 
Quadraturformel nach oben begrenzt ist
\begin{lemma}
    Eine obere Grenzen für die Ordnung einer interpol. Quadraturformel 
    $I^{(n)}$ 
    mit $n+1$ Sützstellen ist $2n+2$
\end{lemma}
\begin{proof}
    Wäre Ordnung höher, könnte man alle Polynome vom Grad $2n+2$ exakt 
    integrieren. Für das Polynom \[
        p(x) \coloneqq \prod_{i=0}^n(x-x_i)^2 \in \Pol_{2n+2}
    \]
    gilt \[
    \forall \igb i 0 n: p(x_i) = 0
    \]
    also \[
        I^{(n)}(p) = 0
    \]
    da \[
        I^{(n)}(p) = \dsum_{j=0}^n w_ip(x_i)
    \]
    aber \[
    \forall x: p(x) \geq 0, \text{ also }p\not\equiv 0
    \]
    demnach \[
        \int_a^b p(x)\diff x > 0 \lightning
    \]    
\end{proof}
Man kann bei geschickter Wahl der Stützstellen also alle Polynome $p\in\Pol_{2n+1}$ exakt integrieren.
Ein Polynom $p\in\Pol_{2n+1}$ kann man immer zerlegen in \[
    p(x) = r(x) \cdot q(x) + s(x)\]
mit $q\in \Pol_{n+1}$ fest vorgegeben, $\deg{q} = n+1$, $r,s\in \Pol_n$
\end{document}