\documentclass[../Skript.tex]{subfiles}
\graphicspath{{\subfix{../images/}}}
\begin{document}
    Numerische Bestimmung/Approximation von Nullstellen (nichtlinearen) Gleichungen wie z.B.
    $$f(x) = 0$$ \[
        a(x) = b(x) \iff a(x) - b(x) = 0\]
    Untersuchung von Minima einer konvexen Funktion $g$, $$f:I\subset \R \to \R\text{ skalar}$$ oder auch
    \[
        f:\Omega \subset \R^n \to \R^n\]\[f:\Omega\subset \C \to \C\]
    \begin{reminder}
        Bei linearen Gleichunge ist die Lösung $x^*$ der Gleichung  $ Ax^* = b$
        bis auf Rundungsfehler ''exakt'' berechenbar (wenn $A$ invertierbar)
    \end{reminder}
    Bei nichtlinearen Gleichung gibt es nur in Ausnahmefällen eine Formel zur Berechnung
    der Lösung, dementsprechend kommen typicherweise iterative Verfahren zur 
    (immer besser werdenden) Approximation der Lösung zum Einsatz.
    Lösung $x^*$ ist dann typicherweise der \underline{Grenzwert} einer iterativ
    erzeugten Folge \(x_{i_{i\in \N}}\) (Konvergenz zeigen{\huge !!!})\\
    Abbruch der Folge nach endlich vielen Versuchen liefert 
    \underline{Näherungslösung} \(\overline{x}\), möglichst mit
    \underline{Abschätzung von \(\dabs{x^*-\overline{x}}\)}\\

    Im einfachsten Fall: \(f:I\subset \R\to\R\) skalare Funktion, I 
    Definitionsintervall von f.
    \[\text{Gesucht: ein }x^*\in I\text{ mit }f(x^*)=0\]
    Analysis: ist \(f:[a,b]\to \R\) \underline{stetig} und \(f(a)\cdot f(b)\leq 0\),
    dann gibt es mindestens eine Nullstelle \(x^*\in [a,b]\) von f (Zwischenwertsatz)
    \\
    Daraus kann man leicht ein Intervallschachtelungsverfahren konstruieren.
% ich mach jetzt :)


\subsection*{Bisektionsverfahren:}
\[[a,b],\ f\colon [a,b]\to \R \text{ stetig, } f(a)\cdot f(b)\leq 0\]
\begin{align*}
    f(a)\cdot f(b)=0 \Rightarrow x*=a \text{ oder } x*=b.   \text{ stop.}\\
    a_0\coloneqq a,\ b_0\coloneqq b, \ i\coloneqq 1\\
\end{align*}
Setze \[x_i\coloneqq\frac{(a_{i-1}+b_{i-1})}{2}\] 
ist \[f(x_i)=0\colon \quad x*=x_i,\ \text{stop.}\]
ist \[b_{i-1}-a_{i-1}\leq \text{TOL, dann stop, } \overline{x}\coloneqq x_i\text{   Damit ist } \overline{x}- x*<\frac{b_{i-1}+a_{i-1}}{2}=\frac{b-a}{2^i}.\]
ist \[f(a_i)\cdot f(x_i)<0 \]
dann\[a_i\coloneqq a_{i-1},\ b_i\coloneqq x_{i}\]
sonst \[a_i\coloneqq x_i,\ b_i\coloneqq b_{i-1}\]
\[\implies i\coloneqq i+1\] 
Setze \[x_i\coloneqq\frac{(a_{i-1}+b_{i-1})}{2}\]
\[\vdots\]
\\
Länge des Intervalls in der \(i\)-ten Iteration:\\
\[\frac{(b-a)}{2^i},\]
Intervallschachtelung (für \(i\to \infty\)) konvergiert gegen genau eine reelle Zahl \(x*\in I\).\\

Damit kann, bei vorgegebener Toleranz TOL, auch die Anzahl der maximal 
(benötigten???????) Iterationen berechnet werden.\\
Verfahren sehr einfach, nutzt nur stetigkeit von \(f\) aus, Konvergiert relativ 
langsam. Wird (auch innerhalb anderer Algorithmen) eingesetzt. (Nur 1D!!!!!!)\\

% leon schreibt <3 baby 
\begin{question}
    Gibt es allgemeiner anwendbare und oder schneller konvergierende Verfahren?
\end{question}
Um solche Algorithmen einfacher untersuchen zu können, stellen wir die Nullpunktsuche 
als Fixpunktiteration dar \[
    f(x^*) = 0 \iff F(x^*) = x^*\]
z.B. mit $F(x) = x - f(x)$, dann gilt nämlich $F(x^*) = x^* - f(x^*) = x^*$
Ausgehend von einem Startwert $x_0$ definiert dies eine Folge $(x_k)_{k\in\N}$ , die im
günstigsten Fall gegen $x^*$ konvergiert, zur Untersuchung solcher Folgen benutzen wir den
Banachschen Fixpunktsatz.
\begin{definition}
    Sei $(M, d)$ ein metrischer Raum, $F:M \to M$ eine Abbildung. Dann heißt $F$ \underline{kontraktion}
    oder kontrahierende Abbildung, wenn es ein $q$ mit $0\leq q<1$ gibt so, dass \[
        d(F(x), F(y)) \leq q\cdot d(x, y) \forall x,y\in M\]  
\end{definition}

%gagagaga gugu

\begin{theorem}[\underline{Fixpunktsatz von Banach}]
    \((M,d)\) ein \underline{vollständiger} metrischer Raum,\\
    \(F:M\to M\) eine Kontraktion mit \(q<1\)\\
    Dann besitzt F genau einen Fixpunkt \(x^*\) in M, d.h. \(\exists !x^*\in M:
    F(x^*)=x^*\).\\
    \(x_0\in M\) beliebiger Startpunkt und \(x_i\coloneqq F(x_{i-1}, i\in N\),
    dann konvergiert \((x_i)_{i\in N}\) gegen \(x^*\) und es gelten die
    Abschätzungen:
    \[\underset{\text{(''a-priori'')}}{d(x^*,x_i)\leq \frac{q^{i}}{1-q}d(x_0,x_1)}
    \] 
    
    Abschätzung der max. notwendigen Anzahl von Iterationen bis auf Toleranz\\
    und
    \[\underset{\text{(''a-posteriori'')}}{d(x^*,x_i)\leq \frac{q}{1-q}d(x_i,x_{i-1})}\]
    In jeder Iteration Fehler abschätzen\\
    
    (Beweis: Analysis, geom. Reihe)
\end{theorem}

Zur ''Geschwindigkeit'' der Konvergenz: \typicherweise normierter Raum \(\R^n\), metrischer Raum \((\Omega,d)\) mit \(\Omega\subset\R^n,\ d(x,y)=\dabs{x-y}
\)
\begin{definition}\hfill\\
Die Folge \((x_i)_{i\in\N}\) mit Grenzwert \(x*\) hat \underline{mindestens die Konvergenzordnung \(p \geq 1\)},
mit \(C<1\) für \(p=1\) , \(C<\infty\) für \(p>1\). \(C\) heißt auch ''Fehlerkonstante''.
\(p=1\): ''lineare Konvergenz'', \(p=2\) ''quadratische Konvergenz''.
\end{definition}
\begin{definition}
    Alternative: \[
    \dabs{x_{i+1}-x_i}\leq C\cdot \dabs{x_i-x_{i-1}}^P,
    \] Äquivalent für Iterationen aus B.F.S \[
    \dabs{F(x_i)-F(x_{i-1})}
    \]
\end{definition}
Folge aus B.F.S.: Konvergenzordnung \(1\) mit \(C=q<1\)

Zunächst weiter skalare Funktionen $f:I\to \R$, Fixpunktfunktion $F:J\to J$, $J\subset I$,
$J$ abgeschlossen. Ist $F$ kontrahierend, so ist $F$ auch stetig. Ist $F$ stetig
differenzierbar, dann lässt sich die Kontraktionszahl über die Ableitung abschätzen
\begin{lemma}
    Sei $F\in C^1([a,b],[a,b])$, dann ist $F$ genau dann kontrahierend auf $[a,b]$, wenn
    $\abs{F'(x)} < 1$ für alle $x\in[a,b]$. 
\end{lemma}
\begin{question}
    Ideen für kontrahierende Abbildung $F$?
\end{question}
Sei $f$ stetig differenzierbar auf $[a,b]$, $f(x^*) = 0$, $x^*\in [a,b]$. Setze 
$F(x) \ceqq x-c\cdot f(x)$ mit $c\in \R\backslash \set{0}$
Ist $F$ kontrahierend? Es gilt \[F'(x) = 1-cf'(x)\]
also \[
    \abs{F'(x)} < 1 \iff \abs{1-cf'(x)} < 1 \iff 0 < cf'(x) < 2\]
dementsprechend muss $c$ klein genug sein und das richtige Vorzeichen haben und $f'(x) \neq 0$.
Man muss das Intervall $J\subset I$ also evtl klein genug um $x^*$ wählen.

\begin{question}
    Was bedeutet dieses $F$ für unsere Fixpunktiteration?
\end{question}
Es gilt \begin{align*}
    x_{i+1} = F(x_i) = x_i - cf(x_i) \iff f(x_i) + \frac{1}{c} (x_{i+1} - x_i)
\end{align*}
Das heißt $x_{i+1}$ ist Nullstelle der Geraden \[
    g(x) = f(x_i) + \frac{1}{c}(x-x_i)\]
Verfeinerung: Wähle $c=c(x) \leadsto \text{ Newton-Verfahren}$




\typicherweise



\end{document}